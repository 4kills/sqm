\documentclass[
	fontsize=12pt,           % Leitlinien sprechen von Schriftgröße 12.
	paper=A4,
	twoside=false,
	listof=totoc,            % Tabellen- und Abbildungsverzeichnis ins Inhaltsverzeichnis
	bibliography=totoc,      % Literaturverzeichnis ins Inhaltsverzeichnis aufnehmen
	titlepage,               % Titlepage-Umgebung anstatt \maketitle
	headsepline,             % horizontale Linie unter Kolumnentitel
	abstract,              % Überschrift einschalten, Abstract muss in {abstract}-Umgebung stehen
]{scrreprt}                  % Verwendung von KOMA-Report

% ------------------------------------------------------------
% LaTeX Template für die DHBW zum Schnellstart!
% Original: https://github.wdf.sap.corp/vtgermany/LaTeX-Template-DHBW
% ------------------------------------------------------------
% ---- Präambel mit Angaben zum Dokument
\input{Inhalt/00_Latex/praeambel}

% ---- Elektronische Version oder Gedruckte Version?
% ---- Unterschied: Die elektronische Version enthält keinen Platzhalter für die Unterschrift
\usepackage{ifthen}
\newboolean{e-Abgabe}
\setboolean{e-Abgabe}{false}    % false=gedruckte Fassung

% ---- Persönlichen Daten:
\newcommand{\titel}{Software-Qualität}
\newcommand{\titelheader}{SQM}
\newcommand{\arbeit}{Hausarbeit}
\newcommand{\studiengang}{Informatik}
\newcommand{\studienjahr}{2023}
\newcommand{\autor}{Dominik Ochs}
\newcommand{\autorReverse}{Ochs, Dominik}
\newcommand{\verfassungsort}{Karlsruhe}
\newcommand{\matrikelnr}{2847475}
\newcommand{\kurs}{TINF20B2}
\newcommand{\bearbeitungsmonat}{Mai 2023}
\newcommand{\abgabe}{XX. Mai 2023}
\newcommand{\bearbeitungszeitraum}{01.10.2022 - XX.05.2023}
\newcommand{\firmaName}{SAP SE}
\newcommand{\firmaStrasse}{Dietmar-Hopp-Allee 16}
\newcommand{\firmaPlz}{69190 Walldorf, Deutschland}
\newcommand{\betreuerFirma}{Christoph Eckert}
\newcommand{\betreuerDhbw}{Dennis Kube \& Jonathan Schwarzenböck}

\input{Inhalt/00_Latex/kopfundFusszeile}

% ---- Hilfreiches
\newcommand{\zB}{z.\,B. }   % "z.B." mit kleinem Leeraum dazwischen (ohne wäre nicht korrekt)
\newcommand{\dash}{d.\,h. }

\newcommand{\code}[1]{\texttt{#1}} % Ist einfacher zu schreiben als ständig \texttt und erlaubt
                                   % Änderungen im Nachhinein, wenn man z.B. Inline-Code anders stylen möchte.

% ---- Silbentrennung (falls LaTeX defaults falsch / nicht gewünscht sind)
\hyphenation{HANA}         % anstatt HA-NA
\hyphenation{Graph-Script} % anstatt GraphS-cript
\hyphenation{Performance-tests}

% ---- Watermark/Wasserzeichen
%\input{Inhalt/00_Latex/watermark} % Auskommentieren wenn nicht erwünscht
%\watermark{gray}{\textbf{DRAFT}}     % Auskommentieren wenn nicht erwünscht. Nimmt optional die opacity/Deckraft z.B. \watermark[0.1]{green}{text} für 10% Deckkraft
%\SetWatermarkSize{8} % Optional. Standard ist 5.8. 

% ---- Beginn des Dokuments
\begin{document}
\setlength{\parindent}{0pt}              % Keine Paragraphen Einrückung.
                                         % Dafür haben wir den Abstand zwischen den Paragraphen.
\setcounter{secnumdepth}{2}              % Nummerierungstiefe fürs Inhaltsverzeichnis
\setcounter{tocdepth}{1}                 % Tiefe des Inhaltsverzeichnisses. Ggf. so anpassen,
                                         % dass das Verzeichnis auf eine Seite passt.
%\sffamily                                % Serifenlose Schrift verwenden.

% ---- Vorspann
% ------ Titelseite
\singlespacing
\include{Inhalt/01_Standard/titelseite}  % Titelseite
\newcounter{savepage}
\pagenumbering{Roman}                    % Römische Seitenzahlen
\onehalfspacing

% ------ Erklärung, Sperrvermerk, Abstact
%\include{Inhalt/01_Standard/sperrvermerk}
%\include{Inhalt/01_Standard/erklaerung}
%\include{Inhalt/02_Abstract/abstract-en}
%\include{Inhalt/02_Abstract/abstract-de}

% ------ Inhaltsverzeichnis
\singlespacing
\tableofcontents

% ------ Verzeichnisse
\renewcommand*{\chapterpagestyle}{plain}
\pagestyle{plain}
%\include{Inhalt/03_Verzeichnisse/abkuerzungen}
\listoffigures                          % Erzeugen des Abbildungsverzeichnisses 
%\listoftables                           % Erzeugen des Tabellenverzeichnisses
\renewcommand{\lstlistlistingname}{Quellcodeverzeichnis}
\lstlistoflistings                      % Erzeugen des Listenverzeichnisses
\setcounter{savepage}{\value{page}}


% ---- Inhalt der Arbeit
\cleardoublepage
\pagenumbering{arabic}                  % Arabische Seitenzahlen für den Hauptteil
\setlength{\parskip}{0.5\baselineskip}  % Abstand zwischen Absätzen
\rmfamily
\renewcommand*{\chapterpagestyle}{scrheadings}
\pagestyle{scrheadings}
\onehalfspacing

\chapter{Einleitung}

Softwarequalitätsmanagement (SQM) ist ein entscheidender Aspekt bei der Entwicklung und Implementierung von Softwarelösungen, 
um sicherzustellen, dass sie effizient, sicher und zuverlässig sind. In dieser Hausarbeit werden wir uns mit dem Vorgehen 
hinsichtlich SQM auseinandersetzen und Optimierungen entwerfen, indem wir ein praktisches Fallbeispiel untersuchen. 
Unser Anwendungsbeispiel ist die Integration einer Recommendation Engine (RE) in eine Lernplattform mit Gamification 
auf dem Hyperscaler Amazon Web Services (AWS). Dieses Beispiel dient dazu, sowohl theoretisches Wissen als auch praktische 
Anwendungsfähigkeiten in Bezug auf SQM nachzuweisen.

Bei meiner Firma, der SAP, gibt es intern eine Lernplattform die Gamification nutzt, um SAP-interne Prozesse 
spielerisch den Kollegen zu vermitteln. Diese Plattform heißt \textit{Digital Heroes}. 
Bisher können die Kollegen die Lernplattform nach konkreten Inhalten durchsuchen. 
Außerdem können die Moderatoren bestimmte Inhalte auf der Startseite featuren. Es gibt jedoch keine 
personalisierten Empfehlungen von Lerninhalten für die Nutzer der Plattform. 
Durch das Einführen einer in-house RE soll sich das nun ändern und damit den Prozess für Nutzer an neue Inhalte 
zu gelangen verbessern.  \\
Das Problem beim Integrieren der RE ist allerdings, dass die bisherige Software dafür angepasst 
und vorbereitet werden muss. Insbesondere arbeitet die RE auf stark-assoziierten Daten, 
was viele sehr aufwendige Joins in der derzeitig genutzten relationalen Datenbank nötig macht, 
was sehr ressourcenintensiv ist.  
Ziel des hier genutzten Anwendungsbeispiels ist es also die RE in die bestehende Software so zu integrieren, 
dass andere Systemteile nicht beeinträchtigt werden. 

Um dieses Ziel zu erreichen, soll das erworbene Wissen aus der Softwarequalität-Vorlesung genutzt werden, 
um das Problem kritisch zu analysieren und zu optimieren. 
Dafür wird zunächst die Ausgangssituation des aktuellen Prozesses untersucht, 
daraufhin wird betrachtet, warum die RE eingeführt wird und warum das bestimmte Anpassungen an der 
bestehenden Software nötig macht. Darauf aufbauend wird beschrieben, wie dadurch der bestehende 
Prozess verbessert werden kann. Weiter wird analysiert, welche Optimierungsmöglichkeiten aus der Sicht 
der SQM es hierfür gibt. Anschließend wird betrachtet, wie der Prozess nach der Optimierung aussieht. 
Schließlich werden in einer Zusammenfassung die grundlegenden Erkenntnisse der Arbeit nochmals bündig dargelegt.

\chapter{Derzeitiger Prozess}

Die gamifizierte Lernplattform namens Digital-Heroes (DH) kann derzeit nach Lerninhalten durchsucht werden 
und es werden \textit{Featured Missions} angezeigt.
Bei dem Durchsuchen nach Lerninhalten kann derzeit eine Volltextsuche verwendet werden. 
Es kann zudem nach groben Kategorien geordnet werden. 
Die Featured Missions, die in \autoref{fig:dh-website} abgebildet sind, 
werden durch das Moderatorenteam ausgewählt -- es steht kein automatischer 
Algorithmus dahinter.

\begin{figure}
	\centering
	\includegraphics[width=1.\textwidth]{Bilder/dh-website.png} 
	\caption{Die Abbildung zeigt die Landing-Page der Digital-Heroes-Lernplattform. 
    Insbesondere lässt sich der Tab der \textit{Featured Missions} erkennen.}
	\label{fig:dh-website}
\end{figure} 

Der Prozess der Auswahl der Featured Missions läuft wie folgt ab: 
\begin{enumerate}
    \item Ein Nutzer oder Moderator erstellt einen neuen Lerninhalt (Mission).
    \item Der neue Inhalt wird von einem Mitglied des Moderatorenteams überprüft (Review), 
    bevor diese Mission veröffentlicht wird.
    \item Ein mal die Woche stimmt sich das Moderatorenteam ab und entscheidet 
    sich für die besten Missionen, die diese Woche erstellt wurden, welche 
    dann in der nächsten Woche gefeatured werden. 
\end{enumerate}
Dieser Prozess und das Fehlen eines automatischen Vorschlagssystems führt 
zu Frustration sowohl aufseiten der Nutzer als auch aufseiten der Moderatoren. 
Die Nutzer würden gerne mehrmals die Woche neue Vorschläge 
(derzeitig die Featured Missions) bekommen und für sie relevante 
Themen direkt vorgeschlagen bekommen, ohne danach explizit suchen zu müssen. \\
Für die Moderatoren ist es dagegen ein großer Arbeitsaufwand 
sich für die Featured Missions zu entscheiden, insbesondere 
ist dabei die Abstimmung mit den anderen Moderatoren zeitaufwendig 
als weiteres Meeting. 

\chapter{Warum eine Recommendation Engine (RE) und was wird dafür nötig?}

Um den Prozess aus dem vorangegangenen Kapitel zu verbessern, 
muss eine automatische Lösung eingebaut werden. \\
In einer benachbarten Abteilung wird eine experimentelle Recommendation Engine (RE) 
namens DANOS entwickelt. Dementsprechend war mein Vorschlag, 
diese für Digital Heroes zu verwenden. Für die entwickelnde Abteilung 
hat dies auch den Vorteil, dass diese frühzeitig schon mit einem praktischen Use-Case 
und einer Success-Story werben können. Für die Nutzer von Digital Heroes 
ermöglicht es, dass sie beliebig viele, genau auf sie zugeschnittene Empfehlungen 
von Lerninhalten bekommen können. Die Moderatoren sparen sich den Aufwand 
der Ermittlung der Featured Missions, da diese ersetzt werden können durch die 
personalisierten Vorgaben. 

Damit die RE integriert werden kann, wird allerdings ein anderer persistenter Speicher nötig.
Grob beschrieben erstellt die RE ihre Empfehlungen für Nutzer A über 
die Präferenzen und bereits absolvierten Missionen von Nutzer B, wenn Nutzer B \textit{ähnlich}
zu Nutzer A ist. Das heißt, um den Rechenaufwand zu verringern, 
soll ein mal berechnet werden, welche Nutzer sich ähnlich sind und die 
Ergebnisse sollen gespeichert werden. Bisher sind alle Inhalte der Lernplattform 
in einer relationalen Datenbank abgelegt. Die RE muss für das Einordnen 
von neuen Inhalten und Nutzern die Ähnlichkeitsbeziehung 
über mehrere Nutzer verfolgen. In einer relationalen Datenbank, 
in dem Nutzer paarweise mit einem Ähnlichkeitsscore abgelegt sind, 
erfordert das Verfolgen der Ähnlichkeit über mehrere Nutzer Datenbankjoins 
derselben Tabelle abhängig von der Länge der Ähnlichkeitskette $n$. 
Bei $m$ Nutzern würde eine Abfrage der Ähnlichkeitskette der Länge $n$ 
eine Tabelle mit $m^n$ Einträgen generieren. Bei bspw. $m = 10000$ und $n=3$ 
bereits $m^n = 10000^3 = 10^{12} = 1~Billion$ Einträge. 

\begin{figure}
	\centering
	\includegraphics[width=1.\textwidth]{Bilder/graph-db-rise-popularity.png} 
	\caption{Anstieg der Popularität verschiedener Datenbankmodelle. (Entnommen aus \cite{db-ranking}.)}
	\label{fig:db-ranking}
\end{figure} 
\chapter{Hauptteil}

\section{Architektur der Applikation}

\begin{figure}
	\centering
	\includegraphics[width=.72\textwidth]{Bilder/architecture.png} 
	\caption{Die Abbildung zeigt die Architektur der Digital Heroes Lernplattform.}
	\label{fig:architecture}
\end{figure} 

Die Architektur der Web-App Digital Heroes mit den verschiedenen verwendeten Technologien
lässt sich in Frontend, Backend und DevOps unterteilen und 
ist in \autoref{fig:architecture} abgebildet. 
Es existieren zwei Instanzen der Architektur: ein Entwicklungs- und ein Produktivsystem. 

\subsubsection*{Frontend:}

Die Frontend-Anwendung der Digital Heroes Web-App ist als Single Page Application (SPA) konzipiert, 
die Vue.js und das Quasar Framework verwendet. 
Vue.js ist ein modernes JavaScript-Framework zum Erstellen von benutzerfreundlichen und reaktiven Webanwendungen.
Quasar ist ein weiteres Framework, das auf Vue.js aufbaut und zusätzliche Funktionen und Komponenten bereitstellt, 
um die Entwicklung von plattformübergreifenden Anwendungen zu erleichtern.
Vue.js nutzt die REST-API des Backends.

\subsubsection*{Backend:}

Das Backend der Web-App besteht aus einer Java-Anwendung, die Eclipse Jersey und Apache Tomcat verwendet. 
Eclipse Jersey ist eine Implementierung der JAX-RS-Spezifikation 
und ermöglicht die einfache Entwicklung von RESTful Web Services für Servlet Container. 
Apache Tomcat ist ein Servlet-Container, der als Webserver zum Bereitstellen von Java-Anwendungen dient
und das Java-Backend hostet.
Als Datenbank kommt die HANA-Datenbank zum Einsatz, die eine high-performance, in-memory relationale Datenbank ist 
und von SAP entwickelt wird. 
Das Java-Backend nutzt das Eclipse-Framework und die HANA-Datenbank um eine REST-API für das 
Vue.js-Frontend bereitzustellen. 

\subsubsection*{DevOps:}

Im Bereich der DevOps-Infrastruktur kommt Jenkins zum Einsatz, 
eine Open-Source-Automatisierungssoftware, 
die zur Implementierung von Continuous Integration und Continuous Deployment (CI/CD) verwendet wird. 
Jenkins ermöglicht die Automatisierung von Build- und Testprozessen sowie die Bereitstellung 
der Anwendung in einer kontrollierten und konsistenten Weise. 
Die Web-App wird auf der SAP Neo Platform-as-a-Service (PaaS) Cloud-Plattform gehostet, 
die eine skalierbare und einfach zu verwaltende Umgebung für die Bereitstellung 
und den Betrieb der Anwendung bietet.
Wird auf den Main-Branch des verwendeten Github-Respositories gepusht, bzw. eine Pull-Request gemerged,
startet Jenkins den CI/CD-Prozess und testet und deployed das neue Backend/Frontend auf der Entwicklungsinstanz. 
Änderungen im Frontend werden zudem direkt auf dem Produktivsystem deployed. 
Für Änderungen des Backends im Produktivsystem kann in SAP Neo ein neuer Release erzeugt werden, 
bei dem das Java-Backend der Entwicklungsinstanz auf das Produktivsystem kopiert wird. 


\section{Ist-Situation und warum die problematisch ist}

Das Entwickler-Team ist auf Deutschland und Australien aufgeteilt, 
was zu Kommunikationsschwierigkeiten und unterschiedlichen Arbeitszeiten führen kann. 
Dies erschwert die Zusammenarbeit und kann die Qualität der Software und die Effizienz 
der Entwicklungsprozesse beeinträchtigen.

Es gibt keine fest vorgegebenen Release-Zyklen; Releases finden nur statt, 
wenn ein neues Feature fertig ist. Dies kann zu unvorhersehbaren und unregelmäßigen Updates führen,
wodurch die Planung und das Management des Projekts schwieriger werden.

Releases werden oft freitags durch das australische Team durchgeführt. 
Dies kann problematisch sein, weil das Team in Deutschland möglicherweise nicht verfügbar ist, 
um eventuell auftretende Probleme sofort zu beheben, was zu längeren Ausfallzeiten führen kann.

Es gibt keinerlei automatisierte Tests. Dies bedeutet, 
dass die Qualität der Software nicht kontinuierlich überprüft wird, 
wodurch Fehler erst spät oder gar nicht entdeckt werden. Dies erhöht das Risiko, 
dass Fehler in die Produktion gelangen, was zu einer schlechteren Benutzererfahrung führt.

Viele Bugs gelangen in die Produktionsumgebung, wodurch oft Hot-Fixes notwendig sind. Dies zeigt, 
dass der Entwicklungsprozess und die Qualitätssicherung unzureichend sind und die Softwarequalität leidet.

Der CI/CD-Prozess für das Deployment auf der Entwicklungsinstanz dauert 15 Minuten. 
Das ist eine lange Wartezeit, die den Entwicklungsprozess verlangsamt und zu einer geringeren Produktivität führt.

Das Backend kann nicht auf dem lokalen Rechner gestartet werden, 
da die HANA-Datenbank nicht lokal gehostet werden kann und kein Mock existiert. 
Dies erschwert die lokale Entwicklung und das Testen, 
was die Effizienz der Entwickler beeinträchtigt und die Qualität der Software gefährdet.

Die Codequalität ist in einem schrecklichen Zustand: Viele duplikative Code-Stücke, 
wenig Struktur und keine Aufteilung der Architektur in Schichten. 
Es wird immer gegen Implementierungen programmiert, anstatt gegen Abstraktionen. 
Dies führt zu einer schlechteren Wartbarkeit, 
erhöht die Fehleranfälligkeit und erschwert die Einarbeitung neuer Teammitglieder.

Der Code ist oft unleserlich, zum Beispiel durch 40-50 Zeilen lange, 
verschachtelte SQL-Statements, die viel Code-Duplikate enthalten. 
Dies führt zu einer schlechteren Wartbarkeit und macht es schwieriger, 
Fehler zu erkennen und zu beheben.

Es gibt keine Code-Reviews, was bedeutet, dass keine systematische Überprüfung der Codequalität 
stattfindet. Dies erhöht die Wahrscheinlichkeit von Fehlern und mindert die Qualität der Software.

Obwohl ein Kanban-Board mit Tasks und Prioritäten verwendet wird, 
gibt es kein Sprint-Planning und kein Sprint-Review. 
Dies führt zu einer schlechteren Planung und Kontrolle der Arbeit 
und beeinträchtigt die Effizienz des Entwicklungsprozesses.

Die Scrum-Meetings finden zweimal pro Woche statt, 
aber mit der ganzen Abteilung (nicht nur Entwickler, sondern auch Designer, Manager, Content-Moderatoren usw.). 
Dadurch werden diese Meetings sehr ineffizient, 
da nicht alle Teilnehmer direkt am Entwicklungsprozess beteiligt sind. 
Dies führt zu einer schlechteren Kommunikation und Koordination innerhalb des Entwicklerteams 
und beeinträchtigt die Effizienz des Entwicklungsprozesses.

Insgesamt sind die identifizierten Probleme im Bereich der Software-Qualitäts-Management 
und Entwicklungsprozesse ein erhebliches Hindernis für die Erreichung einer 
hohen Softwarequalität und einer effizienten Arbeitsweise. Um die Situation zu verbessern, 
müssen die genannten Probleme adressiert und Lösungen gefunden werden, 
die zu einer besseren Organisation, Kommunikation und Qualitätssicherung führen.


% ---- Literaturverzeichnis
\cleardoublepage
\renewcommand*{\chapterpagestyle}{plain}
\pagestyle{plain}
\pagenumbering{Roman}                   % Römische Seitenzahlen
\setcounter{page}{\numexpr\value{savepage}+1}
\printbibliography[title=Literaturverzeichnis]

% ---- Anhang
\appendix
%\include{Inhalt/04_Inhalt/Anhang}
%\clearpage
%\pagenumbering{Roman}  % römische Seitenzahlen für Anhang

\newpage
\end{document}
